\begin{frame}[fragile]
    \frametitle{Back to the Beginning}

    At this point, we have created a basic website with Hugo by creating several Markdown pages. We now need to convert
    these files into HTML, CSS, and JavaScript so that our website can be hosted on the public internet.

    \bigskip

    To convert our markdown files into a website, we simply run the following command in our terminal:\footnote{
        The \texttt{-D} flag instructs Hugo to include Markdown files with \texttt{draft: true} in their header in
        the final compiled website.
    }

    \smallskip

    \begin{lstlisting}[style=saneCode,gobble=8]
        $ ./hugo -D
    \end{lstlisting}

    \bigskip
    The compiled website files can then be found in the \texttt{public} folder in our project directory.
\end{frame}